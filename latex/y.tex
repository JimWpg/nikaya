\documentclass[a4paper,12pt,oneside]{book}
\linespread{1.2}
% \usepackage{hyperref}
\usepackage{xeCJK}
%\punctstyle{quanjiao}
\usepackage{fontspec}
%\usepackage[T1]{fontenc}
%\usepackage[bookmarksopen,bookmarksopenlevel=5]{hyperref}
\usepackage[open,openlevel=1]{bookmark}

\usepackage{indentfirst}
%\usepackage{xeCJK}
\setCJKmainfont{Noto Serif CJK TC}
\setCJKsansfont{Noto Sans CJK TC}
\setCJKmonofont{Noto Sans Mono CJK TC}

\XeTeXlinebreaklocale "zh"
\XeTeXlinebreakskip = 0pt plus 1pt minus 0.1pt

%\xeCJKsetup{PunctStyle=plain}
%\punctstyle{quanjiao}

\usepackage{titlesec} %自定义多级标题格式的宏包
\titleformat{\part}[block]{\Huge\bfseries\centering}{}{1em}{}[]
\titleformat{\chapter}[block]{\huge\bfseries\centering}{}{1em}{}[]
\titleformat{\section}[block]{\Large\bfseries\centering}{}{1em}{}[]
\titleformat{\subsection}[block]{\large\bfseries\centering}{}{1em}{}[]

%\addcontentsline{toc}{section}{蘆葦品}

% \addcontentsline{toc}{subsection}{1 暴流之渡過經}

%\makeatletter

%\newcommand\mysec{\@startsection {section}{1}{\z@}%
%                                   {-3.5ex \@plus -1ex \@minus -.2ex}%
%                                   {2.3ex \@plus.2ex}%
%                                   {\normalfont\Large\bfseries}}
%\makeatother

% endnotes
% pagenotes
% biblatex

\usepackage[hyperref=true,backend=biber,backref=true,style=draft]{biblatex}
%%%\usepackage[superscript]{cite}
\AtBeginBibliography{\small}
\usepackage{wasysym}
\addbibresource{global.bib}

% \let\realcite\cite
% \renewcommand*{\cite}[1]{{\footnotesize\realcite{#1}}}

\newcommand{\mycite}[1]{\textsuperscript{\cite{#1}}}

\begin{document}


\part{有竭篇}
\chapter{1.諸天相應}
\bookmarksetup{open=false}
\bookmarksetupnext{bold=false,italic}
\section{芦苇品}


\subsection{1.暴流之渡過經}
%\label{subsec:sn 1.1}
%\hypertarget{sn 1.1}{target文字} 用来给文字定义带有名称的链接点,


像這樣被我聽聞:\mycite{g1}
%「如是我聞(SA/DA);我聞如是(MA);聞如是(AA)」,南傳作「像這樣被我聽聞」(Evaṃ me sutaṃ,逐字譯為「如是-我-聞」),菩提比丘長老英譯為「這樣我聽到」(Thus have I heard)。「如是我聞……歡喜奉行。」的經文格式,依印順法師的考定,應該是在《增一阿含》或《增支部》成立的時代才形成的(《原始佛教聖典之集成》p.9),南傳《相應部》多數經只簡略地指出發生地點,應該是比較早期的風貌。

有一次,世尊住在舍衛城祇樹林給孤獨園。那時,在夜已深時,容色絕佳的某位天神使整個祇樹林發光後,去見世尊。
抵達後,向世尊問訊,接著在一旁站立。在一旁站立的那位天神對世尊這麼說:

「親愛的先生!你怎樣渡過暴流呢?」

「朋友!我無住立、無用力地渡過暴流。」

「親愛的先生!那麼,依怎樣的方式你無住立、無用力地渡過暴流呢?」

「朋友!當我住立時,那時,我沈沒;朋友!當我用力時,那時,我被飄走,朋友!這樣,我無住立、無用力地渡過暴流。」

「終於我確實看見,般涅槃的婆羅門:

無住立、無用力地,已度脫世間中的執著。」

那位天神說這個,大師是認可者。那時,那位天神[心想]:「大師對我是認可者。」向世尊問訊,然後作右繞,接著就在那裡消失了。



在文档的其他地方,则可以使用命令
%\hyperlink{sn 1.1}{hyperlink} 让另一段文字链接到指定名称的连接点。

\hyperref[subsec:5.1ss]{SN 5.1}


\bookmarksetup{open=true}
\chapter{5.比丘尼相應}

\subsection{1.阿羅毘迦經}
\label{subsec:5.1ss}

被我這麼聽聞:

有一次,世尊住在舍衛城祇樹林給孤獨園。

那時,阿羅毘迦比丘尼在午前時穿好衣服、取衣鉢後,為了托鉢進入舍衛城。

在舍衛城為了托鉢而行後,用餐之後,已從施食返回,獨處欲求者去盲者的樹林。

\clearpage

\printbibliography[heading=bibliography,title=注释]

\end{document}
