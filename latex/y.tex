% !TEX encoding = UTF-8
% !TEX program = lualatex
\documentclass[12pt,oneside]{book}
\usepackage[a4paper,width=150mm,top=25mm,bottom=25mm]{geometry}
\usepackage{fancyhdr}
\usepackage[x11names]{xcolor}
\usepackage[
    pdfauthor={莊春江},
    pdftitle={相應部},
    pdfsubject={佛教},
    pdfkeywords={佛教 南传 上座部 三藏 经藏 巴利},
    %pdfproducer={https://github.com/meng89/nikaya},
    pdfcreator={https://github.com/meng89/nikaya},
    colorlinks=true,
    linkcolor=black,
    citecolor=teal,
    urlcolor=orange,
]{hyperref}

\usepackage[doipre={doi:~}]{uri}

\title{相應部}
\author{莊春江}
\date{2021-12-22}
\linespread{1.2}
\usepackage{fontspec}
%\usepackage[T1]{fontenc}
%\usepackage[bookmarksopen,bookmarksopenlevel=5]{hyperref}
\usepackage[open,openlevel=1]{bookmark}

\usepackage{graphicx}

\usepackage{indentfirst}
\usepackage{luatexja-fontspec}
%\jfontspec 改变当前汉字字体
\usepackage{titlesec} %自定义多级标题格式的宏包

\setmainjfont{Noto Serif CJK TC Light}
\setsansjfont{Noto Sans CJK TC}
%\newjfontfamily{\fontkai}{AR PL UKai TW}
\newjfontfamily{\jfontSansTC}{Noto Sans CJK TC}
\newfontfamily{\fontSansD}{Noto Sans Display}

\usepackage[hyperref=true,backend=biber,backref=true,style=draft]{biblatex}
%%%\usepackage[superscript]{cite}
\AtBeginBibliography{\small}
\usepackage{wasysym}
\addbibresource{global.bib}

% \let\realcite\cite
% \renewcommand*{\cite}[1]{{\footnotesize\realcite{#1}}}

\newcommand{\mycite}[1]{\textsuperscript{\cite{#1}}}
\newcommand{\HRule}{\rule{\linewidth}{0.5mm}}

\pagestyle{fancy}
\fancyhf{}
\rhead{Overleaf}
\lhead{Guides and tutorials}
\rfoot{Page \thepage}



\titleformat{\part}[block]{\fontsize{50}{60}\selectfont \jfontSansTC \bfseries\centering}{}{1em}{}[]
%\titleformat{\chapter}[block]{\huge\bfseries\centering}{}{1em}{}[]
\titleformat{\section}[block]{\Large\bfseries\centering}{}{1em}{}[]
\titleformat{\subsection}[block]{\large\bfseries\centering}{}{1em}{}[]

\setlength{\parindent}{0em}
\setlength{\parindent}{0pt}

\renewcommand{\partname}{}
\renewcommand{\thepart}{}

\begin{document}
\begin{titlepage}
\begin{center}
\null \vspace{2cm}
{\fontsize{130}{140}\selectfont \fontSansTC 相應部}
\\[0.5cm]
{\fontsize{45}{55}\selectfont \fontSansD Saṃyutta Nikāya}
%\HRule
\\[1.5cm]

{\Huge 傳統中文版}
\\ \vspace{0.5em}

\vfill

{\Large 莊春江\ 譯}
\\ \vspace{0.5em}
{\Large 2021-12-31}

\end{center}
\end{titlepage}

\part{有竭篇}

\chapter{諸天相應}
\bookmarksetup{open=false}
\bookmarksetupnext{bold=false,italic}
\section{芦苇品}

\subsection{1.暴流之渡過經}
  像這樣被我聽聞:\mycite{g1}

  有一次,世尊住在舍衛城祇樹林給孤獨園。那時,在夜已深時,容色絕佳的某位天神使整個祇樹林發光後,去見世尊。
抵達後,向世尊問訊,接著在一旁站立。在一旁站立的那位天神對世尊這麼說:

  「親愛的先生!你怎樣渡過暴流呢?」

  「朋友!我無住立、無用力地渡過暴流。」

  「親愛的先生!那麼,依怎樣的方式你無住立、無用力地渡過暴流呢?」

  「朋友!當我住立時,那時,我沈沒;朋友!當我用力時,那時,我被飄走,朋友!這樣,我無住立、無用力地渡過暴流。」

  「終於我確實看見,般涅槃的婆羅門:

  無住立、無用力地,已度脫世間中的執著。」

  那位天神說這個,大師是認可者。那時,那位天神[心想]:「大師對我是認可者。」向世尊問訊,然後作右繞,接著就在那裡消失了。

%在文档的其他地方,则可以使用命令
\hyperlink{sn 1.1}{https://www.google.com} 让另一段文字链接到指定名称的连接点。
%\hyperref[subsec:5.1ss]{SN 5.1}

\href{https://www.google.com}{谷歌}

\subsection{2.解脫經}
  起源於舍衛城。

  那時,在夜已深時,容色絕佳的某位天神使整個祇樹林發光後,去見世尊。抵達後,向世尊問訊,接著在一旁站立。在一旁站立的那位天神對世尊這麼說:

  「親愛的先生!你知道眾生的解脫、已解脫、遠離嗎?」

  「朋友!我知道眾生的解脫、已解脫、遠離。」

  「親愛的先生!那麼,依怎樣的方式你知道眾生的解脫、已解脫、遠離呢?」

  「以有之歡喜的遍盡、以想與識的滅盡、以受的滅與寂靜,朋友!這樣,我知道眾生的解脫、已解脫、遠離。」

\subsection{3.被帶走經}

  起源於舍衛城。

  在一旁站立的那位天神在世尊面前說這偈頌:

  「生命被帶走、壽命是少的,對被帶到老年者來說救護所不存在,

   觀看著這死亡的恐怖,應該作帶來樂的福德。」

  「生命被帶走、壽命是少的,對已被帶到老年者來說救護所不存在,

   觀看著這在死亡上的恐怖,期待寂靜者應該捨棄世間誘惑物。」

\bookmarksetup{open=true}
\chapter{5.比丘尼相應}

\subsection{1.阿羅毘迦經}
\label{subsec:5.1ss}

  被我這麼聽聞:

  有一次,世尊住在舍衛城祇樹林給孤獨園。

  那時,阿羅毘迦比丘尼在午前時穿好衣服、取衣鉢後,為了托鉢進入舍衛城。

  在舍衛城為了托鉢而行後,用餐之後,已從施食返回,獨處欲求者去盲者的樹林。

  那時,魔波旬想要阿羅毘迦比丘尼生出害怕、僵硬、身毛豎立;想要使她從獨處撤退而去見阿羅毘迦比丘尼。抵達後,以偈頌對阿羅毘迦比丘尼說:

  「世間中沒有出離,你以獨處將做什麼?

   請你享受欲、喜樂,不要以後成為後悔者。」

  那時,阿羅毘迦比丘尼這麼想:

  「誰說偈頌?這是人或非人?」

  那時,阿羅毘迦比丘尼這麼想:

  「這是魔波旬,他想要使我生出害怕、僵硬、身毛豎立;想要使我從獨處撤退而說偈頌。」

  那時,阿羅毘迦比丘尼知道:「這是魔波旬。」後,以偈頌回應魔波旬:

  「世間中有出離,被我以慧善觸達,

   波旬!放逸者的親族,你不知道那個境界(足跡)。

   欲如刀叉,諸蘊是它們的砧板,

   凡你所說欲、喜樂者,那對我是不喜樂。」

  那時,魔波旬:「阿羅毘迦比丘尼知道我。」沮喪的、不快樂的,就在那裡消失了。

\part{因緣篇}
\chapter{12.因緣相應}
\section{1.緣起經}

  被我這麼聽聞:

  有一次,世尊住在舍衛城祇樹林給孤獨園。

  在那裡,世尊召喚比丘們:「比丘們!」

  「尊師!」那些比丘回答世尊。

  世尊這麼說:

  「比丘們!我將教導你們緣起,你們要聽它!你們要好好作意!我要說了。」

  「是的,大德!」那些比丘回答世尊。

  世尊這麼說:

  「比丘們!而什麼是緣起呢?比丘們!以無明為緣而有諸行(而諸行存在);以行為緣而有識;以識為緣而有名色;以名色為緣而有六處;以六處為緣而有觸;以觸為緣而有受;以受為緣而有渴愛;以渴愛為緣而有取;以取為緣而有有;以有為緣而有生;以生為緣而老、死、愁、悲、苦、憂、絕望生起,這樣是這整個苦蘊的集,比丘們!這被稱為緣起。

  比丘們!但就以無明的無餘褪去與滅而有行滅(而行滅存在);以行滅而有識滅;以識滅而有名色滅;以名色滅而有六處滅;以六處滅而有觸滅;以觸滅而有受滅;以受滅而有渴愛滅;以渴愛滅而有取滅;以取滅而有有滅;以有滅而有生滅;以生滅而老、死、愁、悲、苦、憂、絕望被滅,這樣是這整個苦蘊的滅。」

  世尊說這個,那些悅意的比丘歡喜世尊所說。

\clearpage

\printbibliography[heading=bibliography,title=注释]

\end{document}
