\documentclass[12pt,a5paper]{book}
\usepackage{longtable}
\usepackage{showframe}
\usepackage{lipsum}
\usepackage{changepage}
% \usepackage{luatexja-fontspec}
\usepackage{fontspec}
\usepackage{xifthen}

%\directlua{
%luaotfload.add_fallback
%("myfallback",
%{
%"Noto Serif:style=Light;"
%}
%)
%}

%[RawFeature={fallback=myfallback}]
\setmainfont{Noto Serif CJK TC ExtraLight}
\setsansfont{Noto Sans CJK TC Light}

%\setmainfont{Noto Serif}
%\setsansfont{Noto Sans}

\usepackage{enumitem}
\usepackage{xcolor}
\usepackage{textcomp}

\newenvironment{Description}
{\list{} {\labelwidth=0pt \itemindent-\leftmargin \let\makelabel\Descriptionlabel \itshape}}
{\endlist}

\newcommand*\Descriptionlabel[1]{\hspace \labelsep  \normalfont \color{blue} \bfseries \sffamily #1}



\newenvironment{EnvNote}[1]{\newcommand{\notenum}{#1} \begin{description}
}{\end{description}}


\newcommand

\newcommand{\subnote}[3][]{
  \ifthenelse{\isempty{#1}}
    {\item[#2]  #3}
    {\item[{\textit#1}#2] #3}
}
\setlist[description]{topsep=10pt, partopsep=0pt, itemsep=-5pt}
\begin{document}
  \begin{description}[labelwidth=\widthof{\bfseries gamma}]

    \item[Kav]   average kinetic energy
    \item[]     number of atoms
    \item[gamma] a Greek letter
  \end{description}

%(1)「生法」(jātidhammaṃ),菩提比丘長老英譯為「屬於生者」(subject to birth)。
%(2)「老法(SA/MA);老之法(AA)」,南傳作「老法」(Jarādhammaṃ),菩提比丘長老英譯為「屬於年老的」(subject to old age)。
%(3)「病法」(byādhidhammo),菩提比丘長老英譯為「屬於生病的」(subject to sickness)。
%(4)「死法」(maraṇadhammo),菩提比丘長老英譯為「屬於死亡」(subject to death)。

  \begin{EnvNote}
    \subnote[1]{「生法」}{ (jātidhammaṃ),菩提比丘長老英譯為「屬於生者」(subject to birth)。}
    \subnote[2]{「老法(SA/MA);老之法(AA)」}{,南傳作「老法」(Jarādhammaṃ),菩提比丘長老英譯為「屬於年老的」(subject to old age)。}
    \subnote[3]{「病法」}{(byādhidhammo),菩提比丘長老英譯為「屬於生病的」(subject to sickness)。}
    \subnote[4]{「死法」}{(maraṇadhammo),菩提比丘長老英譯為「屬於死亡」(subject to death)。}
  \end{EnvNote}



\begin{Description}
  \item[First] The first item
  \item[Second] The second item
  \item[Third] The third etc \ldots
  \item[Fourth]{The fourth item}
  \item[Fifth]{The fifth item, etc.\ldots}
\end{Description}

this is FULL:,s,s,s,
\\
this is HALF:,s,s,s,

\begin{description}
  \item[\textit{(1)}「如是我聞 (SA/DA) ;我聞如是 (MA) ;聞如是 (AA)」],南傳作「像這樣被我聽聞」 (Evaṃ
me sutaṃ,逐字譯為「如是-我-聞」 ),菩提比丘長老英譯為「這樣我聽到」 (Thus have
I heard)。「如是我聞......歡喜奉行。」的經文格式,依印順法師的考定,應該是在《增
一阿含》或《增支部》成立的時代才形成的 ( 《原始佛教聖典之集成》 p.9),南傳《相應
部》多數經只簡略地指出發生地點,應該是比較早期的風貌。
  \item[\textit{(2)}] 多數經只簡略地指出發生地點,應該是比較早期的風貌。
\end{description}

\begin{description}
  \item[\textit{(4)}「如是我聞 (SA/DA) ;我聞如是 (MA) ;聞如是 (AA)」],南傳作「像這樣被我聽聞」 (Evaṃ
me sutaṃ,逐字譯為「如是-我-聞」 sf),菩提比丘長老英譯為「這樣我聽到」 (Thus have
I heard)。「如是我聞......歡喜奉行。」的經文格式,依印順法師的考定,應該是在《增
一阿含》或《增支部》成立的時代才形成的 ( 《原始佛教聖典之集成》 p.9),南傳《相應
部》多數經只簡略地指出發生地點,應該是比較早期的風貌。
  \item[\textit{(5)}] 多數經sf只簡略地指出發生地點,應該是比較早期的風貌。
\end{description}


\begin{description}
  \item[xxx] (1)\lipsum[1] \\
  (2)\lipsum[1]
\end{description}



\begin{longtable}{|p{.05\linewidth} | p{.89\linewidth} |}
\hline
e999 & \lipsum[1] \\ \hline
  888   & bar \\ \hline
foo & bar \\ \hline
foo & bar \\ \hline
foo & bar \\ \hline
foo & bar \\ \hline
foo & bar \\ \hline
foo & bar \\ \hline
foo & bar \\ \hline
foo & bar \\ \hline
foo & bar \\ \hline
\end{longtable}

\lipsum[5]
\end{document}