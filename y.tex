%CTeX 套装最早是由中科院 Leo Wu(吴凌云)研究员开发、维护和发布的 TeX 发行版,
%它是应当时中文配置困难的背景(CCT、CJK 等)而诞生的,而现在的中文支持技术(xeCJK等)配置已非常简单;
%CTeX 套装自 2012 年 3 月 22 日发行 2.9.2.164 版本后,至今未更新,已不能适应当前 TeX 中文技术的发展,
%引用 CTeX 套装的开发之一刘海洋的话:CTeX 已经完成了它的历史使命。

\documentclass[a5paper,12pt]{book}

\usepackage{xeCJK}
%\punctstyle{quanjiao}
\usepackage{fontspec}
%\usepackage[T1]{fontenc}
%\usepackage[bookmarksopen,bookmarksopenlevel=5]{hyperref}
\usepackage[open,openlevel=1]{bookmark}

%\usepackage{xeCJK}
\setCJKmainfont{Noto Serif CJK TC}
\setCJKsansfont{Noto Sans CJK TC}
\setCJKmonofont{Noto Sans Mono CJK TC}

\XeTeXlinebreaklocale "zh"
\XeTeXlinebreakskip = 0pt plus 1pt minus 0.1pt

%\xeCJKsetup{PunctStyle=plain}
%\punctstyle{quanjiao}

\begin{document}

\bookmarksetupnext{open=false}

\part[sort=有]{有偈篇}

\chapter{1 諸天相應}\label{ch:1-諸天相應}


\section{蘆葦品}\label{sec:蘆葦品}

% \addcontentsline{toc}{subsection}{1 暴流之渡過經}
\subsection{1 暴流之渡過經}\label{subsec:1}

像這樣被我聽聞:

有一次,世尊住在舍衛城祇樹林給孤獨園。那時,在夜已深時,容色絕佳的某位天神使整個祇樹林發光後,去見世尊。
抵達後,向世尊問訊,接著在一旁站立。在一旁站立的那位天神對世尊這麼說:

「親愛的先生!你怎樣渡過暴流呢?」

「朋友!我無住立、\ref{subsec:1} 無用力地渡過暴流。」

「親愛的先生!那麼,依怎樣的方式你無住立、無用力地渡過暴流呢?」

「朋友!當我住立時,那時,我沈沒;朋友!當我用力時,那時,我被飄走,朋友!這樣,我無住立、無用力地渡過暴流。」

「終於我確實看見,般涅槃的婆羅門:

無住立、無用力地,已度脫世間中的執著。」

那位天神說這個,大師是認可者。那時,那位天神[心想]:「大師對我是認可者。」向世尊問訊,然後作右繞,接著就在那裡消失了。

\addcontentsline{toc}{chapter}{Chapter 2}
\clearpage

\bookmarksetup{open=true,openlevel=0}% Only open down to chapter
第二篇
\addcontentsline{toc}{part}{第二篇}
This is Chapter 1 of Part 2
\addcontentsline{toc}{chapter}{Chapter 1}
And a section \ref{sec:1}
\addcontentsline{toc}{section}{Section 1 of chapter 1 of part 2}
\clearpage
This is Chapter 2 of Part 2
\addcontentsline{toc}{chapter}{Chapter 2}
\clearpage

第三篇
\bookmarksetup{open=true,openlevel=1}% Open down to section level (i.e. 1)
\addcontentsline{toc}{part}{Part 3}
This is Chapter 1 of Part 3
\addcontentsline{toc}{chapter}{Chapter 1}
\clearpage
This is Chapter 2 of Part 3
\addcontentsline{toc}{chapter}{Chapter 2}
And another section
\addcontentsline{toc}{section}{Section 1 of chapter 2 of part 3}
\clearpage
% Hide again in bookmarks
\bookmarksetup{open=false}
第四篇
\addcontentsline{toc}{part}{Part 4}
This is Chapter 1 of Part 4
\addcontentsline{toc}{chapter}{Chapter 1}
\clearpage
This is Chapter 2 of Part 4
\addcontentsline{toc}{chapter}{Chapter 2}
\clearpage

\end{document}