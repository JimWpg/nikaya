\setuppapersize [A4]

\enableregime[utf]
\mainlanguage[cn]
\language[cn]
\setscript[hanzi] % hyphenation

%%%%%%%
% 字体
\usetypescriptfile[type-imp-myfonts]
\startmode [tc]
  \usetypescript[tcfonts]
  \setupbodyfont[tcfonts,rm,12pt]
\stopmode
\startmode [sc]
  \usetypescript[scfonts]
  \setupbodyfont[scfonts,rm,12pt]
\stopmode
% 字体
%%%%%%%

\usecolors[crayola]
%\usecolors[ral]


%%%%%%%%%%%%
% 页脚页眉
\setupheader
  [text]
  [before={\startframed[frame=off,bottomframe=on,framecolor=black,]},
   after={\stopframed},
  ]
\setupfooter
  [text]
  [before={\startframed[frame=off,bottomframe=off,topframe=on,framecolor=black,]},
   after={\stopframed},
  ]
\setupheadertexts[]
\setupheadertexts[{\tf\bf\ss \somenamedheadnumber{chapter}{current}. \getmarking[chapter]}][]
\setupfootertexts[]
\setupfootertexts[{\ss \doifmodeelse{sc}{点击标题可打开最新修订经文网页}{點擊標題閱讀最新訂正經文}}][{\tfa \it \pagenumber}]
% 页脚页眉
%%%%%%%%%%%


%%%%%%%%%%%%%%%
% PDF 属性信息
\setupinteraction[focus=standard] % 重要功能!添加上后才能连接到位置!
\setupinteraction[state=start,style=normal]
\setupinteraction[
    title={相應部},
    author={莊春江(譯)},
    subtitle={佛教},
    keyword={南傳佛教,尼柯耶,經藏},
    creator={https://github.com/meng89/nikaya/tree/dev},
]
\startmode [sc]
  \setupinteraction[
    title={相应部},
    author={莊春江(译)},
    subtitle={佛教},
    keyword={南传佛教,尼柯耶,经藏},
  ]
\stopmode
% PDF 属性信息
%%%%%%%%%%%%%%%


%%%%%%%%%%%%
% PDF书签
\definelist[tobias]

\placebookmarks[appendix,part,chapter,section,subsection,tobias][chapter][number=no]

\setupinteractionscreen[option=bookmark]
\enabledirectives[references.bookmarks.preroll]


\defineresetset[default][1,0,0][1] %% reset part,  but not chapter, section
\setuphead[sectionresetset=default]

\defineheadplacement[activityhead][vertical]#1#2{%
  \labeltext{\currenthead}\hskip\numberheaddistance #1%
}
% PDF书签
%%%%%%%%%%%


\setuphead[part]      [number=no,alternative=middle,style=\tfd\bf\ss,placehead=yes,header=empty,footer=empty]


%为什么有下面这一大段,是为了自定义标题格式,为了分开number和名字,为了自动给sutta加上 SN.x.x 这样的label
\defineheadalternative
  [centered]
  [alternative=vertical,
   renderingsetup=headrenderings:centered]
\startsetups[headrenderings:centered]
    \vbox {
        \headsetupspacing
        \veryraggedcenter
        \let\\\endgraf
        \let\crlf\endgraf
        \fakeheadnumbercontent
        % \headnumbercontent -> 1.1.1 ,不喜欢这个样式
        \begstrut
        \somenamedheadnumber{chapter}{current}. \headtextcontent
        \endstrut
    }
\stopsetups


\setuphead[chapter]   [alternative=centered, style=\tfc\bf\ss,page=no,header=empty,footer=empty]
\setuphead[section]   [number=no,alternative=middle,style=\tfb\bf\ss]
\setuphead[subsection][style=\tf\bf\ss,]



\define[3]\pian{\part[title=#1(#2-#3)]}
\define[2]\xiangying{\chapter[title=#2, bookmark=#1. #2,]}
\define[3]\pin{\section[title=#1(#2-#3)]}
\define[4]\sutta{\subsection[reference=SN.\somenamedheadnumber{chapter}{current}.#1,
     title={\goto{{\startlua if #1==#2 then context("#1") else context("#1-#2") end \stoplua}. #3}[url(#4)]}]}


% 绿
\define[1]\suttaref{\setupinteraction[color=PineGreen]\goto{#1}[#1]}
% 黄
\define[2]\ccchref{\setupinteraction[color=FuzzyWuzzy]\goto{#1}[url(#2)]}
% 蓝
\define[2]\twnr{\setupinteraction[color=MidnightBlue]\goto{#1}[#2]}


\define[1]\noteTitle{\part[title=#1]}


\define[1]\subnoteref{\textreference[#1]{\null}}

%%%%%%%
%
\define[1]\NoteSubKeyHead{{\it #1}}
\define[1]\NoteKeywordAgamaHead{{#1}}
\define[1]\NoteKeywordNikayaHead{{#1}}

\define[1]\NoteSubEntryKey{{\sl #1}}
\define[1]\NoteKeywordAgama{{#1}}
\define[1]\NoteKeywordNikaya{{#1}}
\define[1]\NoteKeywordBhikkhuBodhi{{#1}}
\define[1]\NoteKeywordBhikkhuNanamoli{{#1}}


\define\newline{\par}

\definestartstop[ReadNote][style=\tfa,]

\definestartstop[Note]
  [style=\small,
      before=\blank\startpacked,
   after=\stoppacked\blank]


\definesymbol [none] []
\defineitemgroup [noteitems]
\setupitemgroup [noteitems][each] [packed] [distance=-10em,symbol=none,inner={\switchtobodyfont[10pt]}]


\definemakeup [homage] [align=middle, style=\tfd,]

%%%%%%
% 附录
\definehead
  [appendix]
  [part]
  [style={\tfb \bf \ss},
      before={},
      after={\blank[3*big]},
      number=no,]

%%%%%%%%%%%%%%%%%%%%%%%%%%%%%%%%%%%%%%%%%%%%%%%%%%%%%%%%%%%%%%%%%%%%%%%%%%%%%%%%%%%%%%%%%%%%%%%%%%%%%%%%%%%%%%%%%%%%%%%%
\startdocument


\startstandardmakeup
\bookmark[tobias]{封面}
  \raggedcenter
  \vfill
  \blank[-3*big]
  \definedfont[SansBold at 140pt]\setstrut \strut \doifmodeelse{sc}{相应部}{相應部}
  \blank[-3*big]
  \definedfont[SansBold at 45pt]\setstrut \strut Saṃyutta ~ Nikāya
  \blank[3*big]
  \definedfont[Sans at 25pt] \setstrut \strut \doifmodeelse{sc}{机转简体版}{傳統中文版}
  \blank[18*big]
  \definedfont[Sans at 20pt] \setstrut \strut 莊春江 ~ \doifmodeelse{sc}{译}{譯}
  \blank[-2*small]
  \definedfont[SansBold at 21pt] \setstrut \strut $date
  \vfill
\stopstandardmakeup


\startstandardmakeup
\part{凡例}
\startReadNote
1.巴利語經文與經號均依「台灣嘉義法雨道場流通的word版本」(緬甸版)。
\blank
2.巴利語經文之譯詞,以水野弘元《巴利語辭典》(昭和50年版)為主,其他辭典或Ven.Bhikkhu Bodhi之英譯為輔,詞性、語態儘量維持與巴利語原文相同,並採「直譯」原則。譯文之「性、數、格、語態」儘量符合原文,「呼格」(稱呼;呼叫某人)以標點符號「!」表示。
\blank
3.註解中作以比對的英譯,採用Ven.Bhikkhu Bodhi, Wisdom Publication, 2012年版譯本為主。
\blank
4.《顯揚真義》(Sāratthappakāsinī, 核心義理的說明)為《相應部》的註釋書,《破斥猶豫》(Papañcasūdaṇī, 虛妄的破壞)為《中部》的註釋書,《吉祥悅意》(Sumaṅgalavilāsinī, 善吉祥的優美)為《長部》的註釋書,《滿足希求》(Manorathapūraṇī, 心願的充滿)為《增支部》的註釋書,《勝義光明》(paramatthajotikā, 最上義的說明)為《小部/經集》等的註釋書,《勝義燈》(paramatthadīpanī, 最上義的註釋)為《小部/長老偈》等的註釋書。
\blank
5.前後相關或對比的詞就可能以「;」區隔強調,而不只限於句或段落。
\stopReadNote
\stopstandardmakeup


\startmakeup[homage]

\doifmodeelse{sc}{\bookmark[tobias]{礼敬世尊}}{\bookmark[tobias]{禮敬世尊}}

\vfill
\blank[-25*big]
$homage
\vfill
\stopmakeup


\input{$suttas}
\page

%%%%%%%%%%%%%%%%%%%%%%%%%%%%%%%%%%%%%%%%%%%%%%%%%%%%%%%%%%%%%%%%%%%%%%%%%%%%%%%%%%%%%%%%%%%%%%%%%%%%%%%%%%%%%%%%%%%%%%%%
% 附录一:注解
\appendix[title={\doifmodeelse{sc}{注解}{註解}}]

\setupheader
  [text]
  [before={\startframed[frame=off,bottomframe=on,topframe=off,framecolor=black,]},
   after={\stopframed},
  ]
\setupfooter
  [text]
  [before={\startframed[frame=off,bottomframe=off,topframe=on,framecolor=black,]},
   after={\stopframed},
  ]

\setupheadertexts[]
\setupheadertexts[{\tf\bf\ss \getmarking[appendix]}][]
\setupfootertexts[]
\setupfootertexts[{\setupinteraction[color=Black]
                   \goto{https://agama.buddhason.org}[url(https://agama.buddhason.org)]}]
                 [{\tfa \it \pagenumber}]

\input{$localnotes}
\input{$globalnotes}
\page


%\appendix[title={\doifmodeelse{sc}{附录二:巴利语经文}{附錄二:巴利語經文}}]
%todo
%\page


\appendix[title={制作说明}]
此佛经来源于\goto{莊春江工作站}[url(https://agama.buddhason.org)],一切相关权利归于译者。
当前是2022年3月份,译者仍在订正经文中。联系译者请前往译者网站留言。
\blank

电子书制作工具也在开发调整当中,请至制作程序项目主页阅读说明以获取制作好的电子书:
\goto{https://github.com/meng89/nikaya}[url(https://github.com/meng89/nikaya)]

\blank
如果打不开上面的链接,请尝试这个云盘链接:

\goto{https://www.jianguoyun.com/p/DbVa44QQnbmtChiojLEE}[url(https://www.jianguoyun.com/p/DbVa44QQnbmtChiojLEE)]

\blank
此电子书制作程序的相关问题请联系我:\goto{me@chenmeng.org}[url(mailto:me@chenmeng.org)]

\stopdocument
