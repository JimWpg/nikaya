\setuppapersize [A4]

\mainlanguage[cn]
\language[cn]
\enableregime[utf]
\setscript[hanzi] % hyphenation

\usetypescriptfile[type-imp-myfonts]
\usetypescript[myfonts]
\setupbodyfont[myfonts,rm,12pt]

\usecolors[ral]

\setupinteraction[state=start]
\setupinteraction[focus=standard] % 重要功能!添加上后才能连接到位置!
\setupinteraction
   [state=start,
    color=RAL5017,
    style=]

\setupurl
   [color=RAL8017,
    style=]

%\setupinteractionscreen[option=bookmark]

\placebookmarks[part,chapter,section,subsection,][chapter][number=no]
\setupinteractionscreen[option=bookmark]
\enabledirectives[references.bookmarks.preroll]


\defineresetset[default][1,0,0][1] %% reset part,  but not chapter, section
\setuphead[sectionresetset=default]

\defineheadplacement[activityhead][vertical]#1#2{%
  \labeltext{\currenthead}\hskip\numberheaddistance #1%
}



\setuphead[part]      [number=no,alternative=middle,style=\tfd\bf\ss,placehead=yes]


%为什么有下面这一大段,是为了自定义标题格式,为了分开number和名字,为了自动给sutta加上 SN.x.x 这样的label
\defineheadalternative
  [centered]
  [alternative=vertical,
   renderingsetup=headrenderings:centered]
\startsetups[headrenderings:centered]
    \vbox {
        \headsetupspacing
        \veryraggedcenter
        \let\\\endgraf
        \let\crlf\endgraf
        %\fakeheadnumbercontent
        % \headnumbercontent -> 1.1.1 ,不喜欢这个样式
        \begstrut
        \somenamedheadnumber{chapter}{current}. \headtextcontent
        \endstrut
    }
\stopsetups


\setuphead[chapter]   [alternative=centered, style=\tfc\bf\ss,page=no]
\setuphead[section]   [number=no,alternative=middle,style=\tfb\bf\ss]
\setuphead[subsection][style=\tf\bf\ss,]



\define[3]\pian{\part[title=#1(#2-#3)]}
\define[2]\xiangying{\chapter[title=#2, bookmark=#1. #2]}
\define[3]\pin{\section[title=#1(#2-#3)]}
\define[3]\sutta{\subsection[reference=SN.\somenamedheadnumber{chapter}{current}.#1,
     title={\startlua if #1==#2 then context("#1") else context("#1-#2") end \stoplua}. #3]}


\define[1]\inbookref{\goto{#1}[#1]}

%\define[2]\cccref{\goto{#2}[url(#1)]}
\define[2]\cccref{\useURL[#2][#1][][#2] \from[#2]}

\define[2]\twnr{\goto{#1}[#2]}

\define[1]\noteTitle{\part[title=#1]}


\define[4]\subnote{\textreference[#1]{#2} {\bf {\bf\it #2}#3}#4}

\define\newline{\par}

\definestartstop
  [Note]
  [before=\blank\startpacked,
   after=\stoppacked\blank]


\startdocument

\input{suttas.tex}

\noteTitle{注解}

\input{notes.tex}

\stopdocument