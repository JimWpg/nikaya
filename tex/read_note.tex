1.巴利語經文與經號均依「台灣嘉義法雨道場流通的word版本」(緬甸版)。
\blank
2.巴利語經文之譯詞,以水野弘元《巴利語辭典》(昭和50年版)為主,其他辭典或Ven.Bhikkhu Bodhi之英譯為輔,詞性、語態儘量維持與巴利語原文相同,並採「直譯」原則。譯文之「性、數、格、語態」儘量符合原文,「呼格」(稱呼;呼叫某人)以標點符號「!」表示。
\blank
3.註解中作以比對的英譯,採用Ven.Bhikkhu Bodhi, Wisdom Publication, 2012年版譯本為主。
\blank
4.《顯揚真義》(Sāratthappakāsinī, 核心義理的說明)為《相應部》的註釋書,《破斥猶豫》(Papañcasūdaṇī, 虛妄的破壞)為《中部》的註釋書,《吉祥悅意》(Sumaṅgalavilāsinī, 善吉祥的優美)為《長部》的註釋書,《滿足希求》(Manorathapūraṇī, 心願的充滿)為《增支部》的註釋書,《勝義光明》(paramatthajotikā, 最上義的說明)為《小部/經集》等的註釋書,《勝義燈》(paramatthadīpanī, 最上義的註釋)為《小部/長老偈》等的註釋書。
\blank
5.前後相關或對比的詞就可能以「;」區隔強調,而不只限於句或段落。