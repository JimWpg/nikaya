\setuppapersize [A4]

\mainlanguage[cn]
\language[cn]
\enableregime[utf]
\setscript[hanzi] % hyphenation

\usetypescriptfile[type-imp-myfonts]
\usetypescript[myfonts]
\setupbodyfont[myfonts,rm,12pt]

\setupinteraction[state=start]
\setupinteraction[focus=standard] % 重要功能!添加上后才能连接到位置!


%\setupinteractionscreen[option=bookmark]

\placebookmarks[part,chapter,section,subsection,][chapter][number=no]
\setupinteractionscreen[option=bookmark]
\enabledirectives[references.bookmarks.preroll]
%\define\empty{}

\setuphead[part]      [number=no,alternative=middle,style=\tfd\bf\ss,placehead=yes]
\setuphead[chapter]   [number=no,alternative=middle,style=\tfc\bf\ss,page=no]
\setuphead[section]   [number=no,alternative=middle,style=\tfb\bf\ss]
\setuphead[subsection][number=no,style=\tf\bf\ss,]



\define[3]\pian{\part[title=#1(#2-#3)]}
\define[2]\xiangying{\chapter[title=#1. #2]}
\define[3]\pin{\section[title=#1(#2-#3)]}
\define[3]\sutta{\subsection[reference=SN.\ctxlua{context(\somenamedheadnumber{chapter}{current} + 1)}.1,
     title={\startlua if #1==#2 then context("#1") else context("#1-#2") end \stoplua}. #3]}

%{\startlua context(SN.\somenamedheadnumber{chapter}{last} + 1.#1) \stoplua}

\define[2]\twnr{\goto{#1}[#2]}
\define[4]\subnote{\reference[#1]{\null} {\bf {\bf\it #2}#3}#4}

\define\note{}

\definestartstop
  [Note]
  [before=\blank\startpacked,
   after=\stoppacked\blank]
%\definestartstop[SubNote]
\definedescription [SubNote]




\startdocument

\pian{有偈篇}{1}{11}
\xiangying{1}{諸天相應}
\pin{蘆葦品}{1}{10}
\sutta{1}{1}{暴流之渡過經}
  \twnr{像這樣被我聽聞}{note.1}:\par
  有一次,世尊住在舍衛城祇樹林給孤獨園。那時,在夜已深時,容色絕佳的某位天神使整個祇樹林發光後,去見世尊。抵達後,向世尊問訊,接著在一旁站立。在一旁站立的那位天神對世尊說這個:\par
  「親愛的先生!你怎樣渡過暴流呢?」


  「朋友!我無住立、無用力地渡過暴流。」


  「親愛的先生!那麼,依怎樣的方式你無住立、無用力地渡過暴流呢?」


  「朋友!當我住立時,那時,我沈沒;朋友!當我用力時,那時,我被飄走,朋友!這樣,我無住立、無用力地渡過暴流。」


  「終於我確實看見,般涅槃的婆羅門:


   無住立、無用力地,已度脫世間中的執著。」


  那位天神說這個,大師是認可者。那時,那位天神[心想]:「大師對我是認可者。」向世尊問訊,然後作右繞,接著就在那裡消失了。


\sutta{2}{99}{解脫經}
起源於舍衛城。
那時,在夜已深時,容色絕佳的某位天神使整個祇樹林發光後,去見世尊。抵達
後,向世尊問訊,接著在一旁站立。在一旁站好後,那位天神對世尊這麼說:
「親愛的先生! 你知道眾生的解脫、已解脫、遠離嗎?」
「朋友! 我知道眾生的解脫、已解脫、遠離。
 」
「親愛的先生! 那麼,依怎樣的方式你知道眾生的解脫、已解脫、遠離呢?」
「以有之歡喜的遍盡、以想與識的滅盡、以受的滅與寂靜,朋友! 這樣,我知道
眾生的解脫、已解脫、遠離。」\par


\subnote{note.1}{\null}{「如是我聞(SA/DA);我聞如是(MA);聞如是(AA)」}{,南傳作「被我這麼聽聞」(Evaṃ me sutaṃ,逐字譯為「如是-我-聞」),
菩提比丘長老英譯為「這樣我聽到」(Thus have I heard)。
「如是我聞……歡喜奉行。」的經文格式,依印順法師的考定,應該是在《增一阿含》或《增支部》成立的時代才形成的(《原始佛教聖典之集成》p.9),
南傳《相應部》多數經只簡略地指出發生地點,應該是比較早期的風貌。}


\page

\xiangying{2}{xx相應}
\pin{xx品}{1}{10}
\sutta{1}{1}{xx經}
xx xx xxx xxx xxx



sfsf\goto{sn.1.1哦}[SN.1.1]

\pian{第二篇}{12}{33}
\xiangying{1}{x相應}
\pin{x品}{1}{10}
\sutta{1}{1}{x經}
slfjsfsfsf


lsfjslfj

\page


\starttext

Let's test!

\startNote
\subnote{note.1}{(1)}{「如是我聞(SA/DA);我聞如是(MA);聞如是(AA)」}{,南傳作「被我這麼聽聞」(Evaṃ me sutaṃ,逐字譯為「如是-我-聞」),
菩提比丘長老英譯為「這樣我聽到」(Thus have I heard)。
「如是我聞……歡喜奉行。」的經文格式,依印順法師的考定,應該是在《增一阿含》或《增支部》成立的時代才形成的(《原始佛教聖典之集成》p.9),
南傳《相應部》多數經只簡略地指出發生地點,應該是比較早期的風貌。}

\subnote{note.1}
{(2)}
{「如是我聞(SA/DA);我聞如是(MA);聞如是(AA)」}
{,南傳作「被我這麼聽聞」(Evaṃ me sutaṃ,逐字譯為「如是-我-聞」),
菩提比丘長老英譯為「這樣我聽到」(Thus have I heard)。
「如是我聞……歡喜奉行。」的經文格式,依印順法師的考定,應該是在《增一阿含》或《增支部》成立的時代才形成的(《原始佛教聖典之集成》p.9),
南傳《相應部》多數經只簡略地指出發生地點,應該是比較早期的風貌。}
\stopNote


\startNote
\subnote{note.1}{(1)}{「如是我聞(SA/DA);我聞如是(MA);聞如是(AA)」}{,南傳作「被我這麼聽聞」(Evaṃ me sutaṃ,逐字譯為「如是-我-聞」),
菩提比丘長老英譯為「這樣我聽到」(Thus have I heard)。
「如是我聞……歡喜奉行。」的經文格式,依印順法師的考定,應該是在《增一阿含》或《增支部》成立的時代才形成的(《原始佛教聖典之集成》p.9),
南傳《相應部》多數經只簡略地指出發生地點,應該是比較早期的風貌。}

\subnote{note.1}{(2)}{「如是我聞(SA/DA);我聞如是(MA);聞如是(AA)」}{,南傳作「被我這麼聽聞」(Evaṃ me sutaṃ,逐字譯為「如是-我-聞」),
菩提比丘長老英譯為「這樣我聽到」(Thus have I heard)。
「如是我聞……歡喜奉行。」的經文格式,依印順法師的考定,應該是在《增一阿含》或《增支部》成立的時代才形成的(《原始佛教聖典之集成》p.9),
南傳《相應部》多數經只簡略地指出發生地點,應該是比較早期的風貌。}
\stopNote


\stoptext


\stopdocument